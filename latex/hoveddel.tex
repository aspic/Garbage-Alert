\section{Hoveddel}
2 til 4 situasjonsfortellinger
Teori og begreper flettes inn underveis
Refleksjoner rundt situasjonene, både individ- og gruppenivå
Aksjoner (tiltak/videreføre, reflektere)


\subsection{Gruppens kommunikasjonsmønster} %gruppelogg 08.02
Under en av gruppens diskusjoner 8. februar fikk vi en tilbakemleding av fasilitator på hvordan kommunikasjonen i gruppen utartet seg. For å gå inn i dybden på dette kan det nevnes hvordan stemningen på gruppen var på starten av dagen. Forrige onsdag fikk vi utviklet en idè som alle på gruppen var fornøyde med. Etter flere timers arbeid med dette hadde vi en oppsummering sammen med landsbyens leder for å oppdatere henne på hvordan vi lå an. Det viste seg at idèen vår ikke var bra nok med tanke på at resirkulering ikke stod som hovedfokus. For å følge opp problemsstillingen vår og landsbymålene måtte vi gjøre store endringer. Dette ble en dårlig avslutning på dagen, og motivasjonen sank betraktelig. 

Da vi møtte opp onsdag 8. februar var den generelle stemningen på gruppen at medlemmene var demotiverte. Det at idèen vår fra den foregående onsdagen føltes som bortkastet lå fortsatt i tankene til alle på gruppen. Kjetil nevnte at han var lite klar for å gå tilbake et skritt og gå gjennom en helt ny idèmyldring, ettersom vi forrige gang endelig følte at vi hadde en idè som alle var fornøyde med. Dette utsagnet var ikke akkurat noe som bidro til en bedre start på dagen for resten av gruppen. Trond og Andreas tenkte at man bare må gjøre det beste utav det, og de startet en diskusjon som gikk ut på å dekomponere idèen vår fra forrige gang. Kjetil kastet seg for på, og motivasjonen hans økte når han innså at dette kanskje ikke var verdens undergang. Etter en tid med brainstorming videreutviklet og endret vi det vi jobbet med forrige onsdag. Lydnivået økte og motivasjonen på gruppen steg blandt flere av medlemmene. En ny idè ble født.

Da Andreas følte at gruppen virkelig var i sving, kom fasilitator bort til gruppen med en skisse som viste hvordan kommunikasjonen på gruppen var.
%bilde av kommunikasjonsmønster på gruppen
Dette kom som en overaskelse på gruppen. Det viste seg at Andreas, Kjetil og Trond var mest aktive, mens Ina og Christian var mer passive. Dette måtte diskuteres i gruppen.
Ina mener at årsaken til hennes passivitet er i tråd med hennes vanlige oppførsel under et gruppearbeid. Hun sier at hun er meget observant og enig i det som skjer, men at hun ikke alltid legger merke til at hun kan virke passiv for resten av gruppen. Når hun er enig tenker hun ikke over å si dette til resten av gruppen, eller gi de andre hennes godskjennelse. Hvis hun derimot er uenig sier hun ifra. Hun understreker at hun ikke er tilbakeholdende eller ukomfortabel med å si ting foran hele gruppen. Ina sier at hun føler seg like aktiv som de andre selv om hun snakker mye mindre. Dette er derimot ikke resten av gruppens oppfatning, som i dette tilfelle oppfattet henne som passiv. 

Kjetil mener at man henvender seg til de som er mest aktive, ettersom man får mer respons fra dem. Andreas sier at grunnen til at vi ikke inkluderer alle ikke har noe med at man trenger “godkjenning” fra enkelte av gruppenes medlemmer, men at diskusjonen hoper seg opp der folk er mest aktive. Dette observerte Andreas senere på dagen, da mønsteret flyttet på seg. 

Christian var veldig trøtt på morgenen, og merket at han var litt i kjelleren. Etter en kaffekopp ble han mer sentral i diskusjonen. Etter at vi ble observante på hvordan dette mønsteret var, tok vi en pause; for å tenke litt på andre ting, ta en kaffe, og reflektere litt over hvorfor det var slik. 

Det var ingen tvil om at aktiviteten på gruppen ble mer balansert etter denne pausen. Vi ble klar over hvem vi henvender oss til når vi snakker. Andreas sa senere at han begynte å ta dette i betrakning i senere diskusjoner, der han beveger blikket mer for å få visuell kontakt med alle på gruppen. Trond foreslå at man kan legge til spørsmål til gruppemedlemmene om de er enige eller uenige, og rette disse mot dem på gruppen som er mer passive. Dette bidrar til å få mer kommunikasjon mellom alle på gruppen, og inkluderer dem som kan være umotiverte og trøtte eller ikke kommuniserer like mye. 

Gruppen ble også bevisst på et annet problem under videre diskusjon om spillidèen. Idèen utviklet seg raskt og det var stor enighet i gruppen. Men da vi begynte å tegne og visualisere idèen viste det seg at alle så for seg forskjellige ting. Selv om kommunikasjonen så ut til å fungere etter situasjonen om gruppens kommunikasjonsmønster, var det likevel store forskjeller i hva de forskjellige medlemmene trodde vi diskuterte. Å gå mer i dybden på det vi diskuterer, snakke om detaljer og være klar over at folk tenker forskjellig førte videre til at slike misforståelser blir unngått.

%johnson, hvordan gjøre mangfold til noe positivt
%teori, forskjellige roller i en gruppe?
 \cite{Johnson}