\section{Presentasjon av gruppen}
``En spillverden på ett semester`` viste seg å være attraktiv for
studenter som studerer datateknikk og informatikk, så det kom ikke some
noe overraskelse at 3/5 av gruppa hadde en slik spesialisering.

\subsection{Forventninger og ambisjoner}
I løpet av de første landsbydagene gikk flere diskusjoner ut på hvilke
ambisjoner og forventinger vi hadde til Eksperter i Team, gruppen og
ikke minst resultatet. Det var samstemt enighet om at vi skulle gå for
en god karakter.

Kjetil hadde delte meninger om Eksperter i Team. Desse meningene var
hovedsaklig basert på rykter og fortellinger fra andre studenter. Noen
mente at dette kurset var et tidssluk og ``svada'', andre sa at det var
et av de bedre kursene på NTNU. Det at denne landsbyen var førstevalget
hans var dog en motivasjonsfaktor.
\\
\\
\textbf{Andreas Røysland Aarnes}\\
Andreas er 25 år gammel, fra Stavanger og studerer geologi og geofysikk. Han jobber for Det Norske Oljeselskap og driver ellers mye med musikk. Innen spillverdenen er han spesielt interessert i strategispill. EiT for Andreas viste seg å være mye kjekkere enn det han hadde forventet, ettersom han får bruke sine kreative og sosiale sider på nye måter. 

Gruppen ser på Andreas som en optimistisk og entusiastisk person. Han er
ivrig og sosial, noe som gir ham en stor innflytelse på gruppen. En av
hans svake sider er at hans entusiasme og innflytelse kan ha en negativ
innvirkning på gruppens avgjørelser, ettersom han kan virke noe
dominerende. \\\\
\textbf{Christian Aleksander Lysne}\\
Christian er 22 år, født og oppvokst i Trondheim. Han studerer prosjektledelse med spesialisering i industriel økonomi, og er tidligere utdannet som dataingeniør. På fritiden spiller Christian basketball og jobber med spillfirmaet sitt, Peekaboo. Det er denne spillbakgrunnen som var avgjørende for at Christian valgte denne EiT- landsbyen. Han hadde ikke forventet å lære noe nytt på EiT, men ble positivt overrasket.   

Gruppens syn på Christian er at han virker som en rolig person som
alltid har kontroll. Han blir fort engasjert av andres inspill og er
flink til å lytte til de andre på gruppen. Enkelte av gruppemedlemmene
mener Christian er en typisk lærer/leder; en man henvender seg til fordi
han alltid har kontroll og er hjelpsom. Christian mener selv at han har
en tendens til å kjede seg når han jobber med det samme lenge. Han liker
variert arbeid. \\\\
\textbf{Ina Sander Pedersen}\\
flink å lage muffins ... (Ina må skrive litt om seg selv her)

Ina ser på seg selv som ryddig og presis i sitt arbeid, noe resten av
gruppen er enig i. Gruppen synes at Ina er en rolig person som er mer
forsiktig enn impulsiv. Men når en diskusjon begynner å bli ferdig og
gruppemedlemmene begynner å bli enige, er Ina flink til å kartlegge og
få kontroll over situasjonen og hva gruppen har kommet frem til. Gruppen
synes hun er vennlig og aksepterende.\\\\
\textbf{Kjetil Mehl}\\
23 år gammel datateknikkstudent fra Rosendal. Har jobbet en del med
spill tidligere. Han hadde en positiv innstilling til EiT, men var
redd det kunne ta noe mye tid.

Gruppen synes Kjetil er flink til å ha kontroll over arbeidet underveis
i prosessen. Han er objektiv og analytisk, i tillegg til å være en
person som holder en god stemning på gruppen. Han jobber veldig
effektivt når han jobber alene, men det hender at han kan bli litt
fraværende og usosial når han først er godt i gang med individuelt
arbeid. \\\\
\textbf{Trond Kjetil Bremnes}\\
He is from the far north! \\

\subsection{Samarbeidsavtalen}
En samarbeidsavtale~\cite{samarbeidsavtale} er en skriftlig avtale
inngått mellom alle parter i en samarbeidskontekst. Denne avtalen skal
danne rammer for hva de ulike partene kan forvente av samarbeidet, og
hva de eventuelt ikke kan forvente. Denne avtalen er retningslinjer for
samarbeidet, og skal nødvendigvis ikke brukes for å ``straffe''
medlemmer som bryter avtalen.

Under den første landsbydagen fikk gruppen i oppgave å forfatte en
samarbeidsavtale. Arbeidet med å utrede denne avtalen gikk veldig fint,
og det var få uenigheter. Dette styrket i alle fall Kjetil sitt inntrykk
av at gruppen var veldig samstemt, selv om medlemmene hadde forskjellige
studiebakgrunner.\\
\\
 De viktigste punktene samt hva vi ble enige om er tatt
med nedenfor:
% Ta med her?
\begin{itemize}
	\item Hovedmomenter: Gruppen skal sette felles mål, og alle skal
	være klar over desse.  Gruppa skal være åpen for at alle skal komme med
	innspill.
	\item Møtestruktur: Hver onsdag skal vi starte dagen med å fortelle
	hva man har gjort siden sist, i tillegg til hva man skal gjøre/har som
	mål i løpet av dagen. Her er det også viktig å belyse eventuelle
	problemer en har i plenum med resten av gruppa.
	\item Beslutningsstrategi: Er det noe vi absolutt ikke blir enige
	om, vil flertallet bestemme.
\end{itemize}
Det å fastsette en samarbeidsavtale så tidlig i prosessen viste seg å
være en god gruppeøvelse. Det gjorde at alle medlemmene måtte ytre hva
de ønsket å oppnå og forventet av Eksperter i Team. Dette førte til at
medlemmene ble litt tryggere på hverandre, og fekk noen felles
``rammer''. Gitt at uenigheter og/eller gjentagende uheldige hendelser
oppstår, kan vi peke på samarbeidsavtalen og aksjonere ut fra den.
