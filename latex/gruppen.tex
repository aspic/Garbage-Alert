\section{Presentasjon av gruppen}
``En spillverden på ett semester`` viste seg å være attraktiv for
studenter som studerer datateknikk og informatikk, så det kom ikke some
noe overraskelse at 3/5 av gruppa hadde en slik spesialisering.

\subsection{Forventninger og ambisjoner}
I løpet av de første landsbydagene gikk flere diskusjoner ut på hvilke
ambisjoner og forventinger vi hadde til Eksperter i Team, gruppen og
ikke minst resultatet. Det var samstemt enighet om at vi skulle gå for
en god karakter.

Kjetil hadde delte meninger om Eksperter i Team. Desse meningene var
hovedsaklig basert på rykter og fortellinger fra andre studenter. Noen
mente at dette kurset var et tidssluk og ``svada'', andre sa at det var
et av de bedre kursene på NTNU. Det at denne landsbyen var førstevalget
hans var dog en motivasjonsfaktor.
\\
\\
\textbf{Andreas Røysland Aarnes:}\\
Ein ivreg student frå Stavangers beste vestkant. Likar turar i fjellet
og hjemmelaga pizza.\\
\textbf{Christian Aleksander Lysne:}\\
\textbf{Ina Sander Pedersen:}\\
\textbf{Kjetil Mehl:}\\
23 år gammel datateknikkstudent fra Rosendal. Har jobbet en del med
spill tidligere. Han hadde en positiv innstilling til EiT, men var
redd det kunne ta noe mye tid.\\
\textbf{Trond Kjetil Bremnes:}\\

\subsection{Samarbeidsavtalen}

