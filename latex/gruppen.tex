\section{Presentasjon av gruppen}
``En spillverden på ett semester`` viste seg å være attraktiv for
studenter som studerer datateknikk og informatikk, så det kom ikke some
noe overraskelse at 3/5 av gruppa hadde en slik spesialisering.

\subsection{Forventninger og ambisjoner}
I løpet av de første landsbydagene gikk flere diskusjoner ut på hvilke
ambisjoner og forventinger vi hadde til Eksperter i Team, gruppen og
ikke minst resultatet. Det var samstemt enighet om at vi skulle gå for
en god karakter.

Kjetil hadde delte meninger om Eksperter i Team. Desse meningene var
hovedsaklig basert på rykter og fortellinger fra andre studenter. Noen
mente at dette kurset var et tidssluk og ``svada'', andre sa at det var
et av de bedre kursene på NTNU. Det at denne landsbyen var førstevalget
hans var dog en motivasjonsfaktor.
\\
\\
\textbf{Andreas Røysland Aarnes:}\\
Ein ivreg student frå Stavangers beste vestkant. Likar turar i fjellet
og hjemmelaga pizza.\\
\textbf{Christian Aleksander Lysne:}\\
\textbf{Ina Sander Pedersen:}\\
\textbf{Kjetil Mehl:}\\
23 år gammel datateknikkstudent fra Rosendal. Har jobbet en del med
spill tidligere. Han hadde en positiv innstilling til EiT, men var
redd det kunne ta noe mye tid.\\
\textbf{Trond Kjetil Bremnes:}\\

\subsection{Samarbeidsavtalen}
Under den første landsbydagen fikk gruppen i oppgave å forfatte en
samarbeidsavtale~\cite{samarbeidsavtale}. Målet med denne avtalen var å
lage noen rammer rundt gruppesamarbeidet som alle i gruppen var
inneforstått med. De viktigste punktene samt hva vi ble enige om er tatt
med nedenfor:
% Ta med her?
\begin{itemize}
	\item Hovedmomenter: Gruppen skal sette felles mål, og alle skal
	være klar over desse.  Gruppa skal være åpen for at alle skal komme med
	innspill.
	\item Møtestruktur: Hver onsdag skal vi starte dagen med å fortelle
	hva man har gjort siden sist, i tillegg til hva man skal gjøre/har som
	mål i løpet av dagen. Her er det også viktig å belyse eventuelle
	problemer en har i plenum med resten av gruppa.
	\item Beslutningsstrategi: Er det noe vi absolutt ikke blir enige
	om, vil flertallet bestemme.
\end{itemize}
Det å fastsette en samarbeidsavtale så tidlig i prosessen viste seg å
være en god gruppeøvelse. Det gjorde at alle medlemmene måtte ytre hva
de ønsket å oppnå og forventet av Eksperter i Team. Dette førte til at
medlemmene ble litt tryggere på hverandre, og fekk en felles
``platform''.
