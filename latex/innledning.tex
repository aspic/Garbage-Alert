\section{Innledning}
Målet i Eksperter i Team er at studentene skal anvende sin fagkompetanse
og utvikle sin samspillkompetanse gjennom å arbeide resultatrettet med
relevante problemstillinger og situasjoner. Nedenfor er den overordnede
intensjonen av Eksperter i Team gjengitt:\\

``Målet med Eksperter i Team (EiT) er at studentene både skal anvende
sin fagkompetanse og utvikle sin samspillskompetanse gjennom å arbeide
resultatrettet med relevante problemstillinger for samfunns- og
arbeidsliv. Studentene skal utvikle sine teoretiske kunnskaper og
praktiske ferdigheter i tverrfaglig prosjektarbeid og få trening i
yrkesrelevante arbeidsmåter''\\

Studentene skal gjennom felles refleksjon vinne økt innsikt i
handlingsmønstre og væremåter som kreves for å få et godt resultat i et
tverrfaglig prosjektarbeid. Emnet skal bidra til økt innsikt i andre
fags egenart og deres måter å arbeide på. Samtidig skal emnet bidra til
å styrke egen faglig identitet, gjennom samspillet i gruppen og måten
egen fagkunnskap bidrar til prosjektet på.``
