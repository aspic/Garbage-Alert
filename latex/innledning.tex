\section{Innledning}
Målet i Eksperter i Team er at studentene skal anvende sin fagkompetanse
og utvikle sin samspillkompetanse gjennom å arbeide resultatrettet med
relevante problemstillinger og situasjoner. Nedenfor er den overordnede
intensjonen av Eksperter i Team gjengitt:\\

``Eksperter i team er et yrkesforberedende emne som lærer studentene å
samarbeide gjennom å anvende sin fagkunnskap i et tverrfaglig
prosjektarbeid. Aktuelle problemområder fra samfunns- og arbeidsliv
danner utgangspunktet for prosjektarbeidet, og studentgruppene bør
samarbeide med eksterne partnere. Studentenes samlede fagkunnskap må
være tilpasset prosjektet gruppen skal arbeide
med.\\

Studentene skal gjennom felles refleksjon vinne økt innsikt i
handlingsmønstre og væremåter som kreves for å få et godt resultat i et
tverrfaglig prosjektarbeid. Emnet skal bidra til økt innsikt i andre
fags egenart og deres måter å arbeide på. Samtidig skal emnet bidra til
å styrke egen faglig identitet, gjennom samspillet i gruppen og måten
egen fagkunnskap bidrar til prosjektet på.``
