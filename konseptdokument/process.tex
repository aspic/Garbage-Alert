\section{Prosessen}
Dette avsnittet forteller hvordan konseptet ble til ved å gå fra en
problemstilling, til en idè og til slutt et konsept. Den første delen av
denne prosessen var ikke iterativ, ettersom gruppen i praksis gikk fra
en vagt definert problemstilling, til en idè, for så å redefinere ideen
for å bedre passe den valgte problemstillingen. Prosessen gikk deretter
fra idè til konsept.
\subsection{Fra problemstilling til idè}
Det ble brukt mye tid på å utforske forskjellige ideer som var
interessante. Gruppen følte seg ikke veldig bundet av den opphavlige
problemstillingen, og eksperimenterte derfor med forskjellige ideer.
Gitt problemstillingen i Avsnitt~\ref{sec:problemstillingen} var det
naturlig å tenke på et spill som var turbasert eller sanntidsspill med
flere spillere. Etter flere runder internt ble det enighet rundt
følgende idè.
\begin{description}
\item[Energispillet med flerspillermodus] \hfill\\
Spillere skal organisere hver sin lille by, hvor fokuset er å forhandle
med hverandre på grunnlag av hvilke energitype de forskjellige byene
velger å gå for. En by som for eksempel fokuserer på kull kan raskt få
mye resurser, men ville påvirke sitt eget miljø lokalt, og senere hele
miljøet globalt.
\end{description}
Etter tilbakemeldinger fra landsbylederen viste deg seg derimot at ideen
ikke samsvarte godt nok med problemstillinga som var valgt. Dette førte
til restrukturering av den opphavlige ideen, for å få mer fokus på
gjenbruk og resirkulering. Følgende elementer fra ideen ble tatt med
videre:
\begin{itemize}
	\item Flerspiller
	\item Hver spiller har en base
	\item Bruk av søppel som ressurser, istedenfor energi
	\item Lokale faktorer kan påvirke det globale miljøet
\end{itemize}
\subsection{Redefinering av ideen}
Den opphavlige ideen skle over på å bli et rent konkurransespill, med
mindre fokus sanking/utvinning av resurser. For å minske ressursfokuset
ble det bestemt at hver øy er en ``søppeløy'', og målet til spilleren er
å fjerne alt søppelet på sin øy.  Ressurser blir utvinnet ved å
resirkulere søppelet på øya. Ved å legge inn denne vrien ble ideen mer
konformt med problemstillingen. 
\begin{description}
\item[Garbarge Alert] \hfill\\
Spillerene skal organisere hver sin lille søppelfylt øy. Målet er å
fortest mulig kvitte seg med søppelet på sin øy. Dette kan bli gjort ved
resirkulering, eller ved å sabotere for de andre spillerene ved å flytte
ditt søppel over til dem. Utifra aksjonene til de forskjellige
spillerene kan både lokale og globale katastrofer inntreffe. Spilleren
som først rydder sin øy, eller som står igjen til sist vinner spillet.
\end{description}
Utifra denne korte beskrivinga begynte vi å arbeide opp mot et konkret
konsept, som beskrevet i Avsnitt~\ref{sec:konsept}.
%Det ble gjort koblinger opp mot et spill lignende Advanced
%Wars\cite{advancewars} eller Wordfeud\cite{wordfeud} med fokus på
%miljøet.
