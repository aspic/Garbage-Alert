\section{Prosessen}
Dette avsnittet forteller hvordan konseptet ble til ved å gå fra en
problemstilling, til en idè og til slutt et konsept. Den første delen av
denne prosessen var ikke iterativ, ettersom gruppen i praksis gikk fra en
vagt definert problemstilling, til en idè, for så å redefinere
problemstillingen litt for å bedre passe ideen. Prosessen gikk deretter
fra idè til konsept.
\subsection{Fra problemstilling til idè}
Det ble brukt mye tid på å utforske forskjellige ideer som var
interessante. Gruppen følte seg ikke veldig bundet av den opphavlige
problemstillingen, og eksperimenterte derfor med forskjellige ideer.
Gitt problemstillingen i Avsnitt~\ref{sec:problemstillingen} var det
naturlig å tenke på et spill som var turbasert eller sanntidsspill med
flere spillere. Etter flere runder internt ble det enighet rundt
følgende idè.
\begin{description}
\item[Energispillet med flerspillermodus] \hfill\\
Spillere skulle organisere hver sin lille by, hvor fokuset var å
forhandle med hverandre på grunnlag av hvilke energitype de forskjellige
byene valgt å gå for. En by som for eksempel fokuserte på kull kunne
raskt få mye resurser, men ville påvirke sitt eget miljø lokalt, og
senere hele miljøet globalt.
\end{description}
Etter en tilbakemeldinger fra landsbylederen viste deg seg derimot at
ideen ikke samsvarte godt nok med problemstillinga som var valgt. Dette
førte til restrukturering av den opphavlige ideen, for å få mer fokus på
gjenbruk og resirkulering. Følgende elementer fra ideen ble tatt med
videre:
\begin{itemize}
	\item Flerspiller
	\item Hver spiller har en base
	\item Bruk av søppel som resurser, istedenfor energi
	\item Lokale faktorer kan påvirke det globale miljøet
\end{itemize}


%Det ble gjort koblinger opp mot et spill lignende Advanced
%Wars\cite{advancewars} eller Wordfeud\cite{wordfeud} med fokus på
%miljøet.
