\section{Konseptet}\label{sec:konsept}

Målet med dette avsnittet er å gi et innblikk i det utviklede konseptet,
hvilke spillaspekter som er tatt med og spilldynamikken.

\subsection{Garbage Alert}

			\begin{figure} [here]
				\begin{center}
					\includegraphics[scale=0.5]{splashscreen}
				\end{center}
			\caption{Garbage Alert}
		\end{figure}

\subsection{Introduksjon}

Spillerene skal organisere hver sin lille søppelfylte øy. Målet er å
fortest mulig kvitte seg med søppelet på sin øy. Dette kan bli gjort ved
resirkulering, eller ved å sabotere for de andre spillerene ved å flytte
ditt søppel over til dem. Utifra aksjonene til de forskjellige
spillerene kan både lokale og globale katastrofer inntreffe. Spilleren
som først rydder sin øy, eller som står igjen til sist vinner spillet.


\subsection{Spillanalyse}

Spillbeskrivelse: Spillets sjanger er valgt til flerspiller sanntidsstrategi, men det vil også være mulighet for å spille alene mot datamaskinen. Viktige spillelementer vil være samling, ressursforvaltning, valg av våpen og forsvar, samt timing av våpenbruk. Spillinnholdet vil hovedsaklig være Pure Play (??), der målet er å få spilleren til å ha det gøy, men det vil også være elementer av realisme da det spilleren foretar seg vil virke inn på både lokalt og globalt miljø. Det vil være en lineær spillsekvens (Er usikker på denne fordi jeg ikke finner noen  spillsekvenstype som passer??) der spilleren får tilgang til nye våpentyper og forsvarsmuligheter etterhvert som krav for hver enkelt oppgradering er nådd. For at spillet skal få et lekent og fristende utseende er det valgt en type tegneseriestil som vil gjøre nettopp dette. 

\subsection{Spillatmosfære}

This is where it is best to have a mood board or a clear description of the game’s style. 

This is a good place to start interacting with a graphic designer.\\

Atmosphere Mood Board\\
Character  Units Sketch and Description\\
A Level(Locations) Sketch and Description\\
Audio Description

\subsection{Game Play}

Hver spiller blir tildelt en øy dekket med søppel. En øy består av en
gjenvinningsstasjon, tre våpenstasjoner og en forsvarssone.
Gjenvinningsstasjonen har som oppgave å transformere øyas søppel om til
ressurser. Disse ressuresene kan brukes til å kjøpe våpen,
forsvarsmurer, utvinne nye ressurser og oppgradere
gjennvinningsstasjonen. Mulige oppgraderinger for gjenvinningsstasjonen
er raskere gjennvinning og større kapasitet for inntak av søppel. Våpen
brukes til å skyte søppel på motstandernes øyer, og forsvarsmuren har
som oppgave å blokkere slike angrep. 

Det er flere muligheter for å vinne i dette spillet. Det første
vinnsenarioet er å bli den første til å kvitte seg med alt søppelet på
øyen sin. Den andre måten å vinne på er å skyte i stykker motstandernes
gjenvinningsstasjon. Hvis en øy mister gjenvinningsstasjonen sin, vil
den bli oversvømt med søppel og synke ned i havet. Dermed kan spilleren
velge om han/hun vil fokusere på å forbedre sin egen øy eller ødelegge
for andre. \\
\\
Using this outline to create a descriptive paragraph about how the game is played. 

The idea is that you want the person imagine they are actually playing the game.

Do not use Generic names when writing about the game play. 

Example: No one wants to here that enemy 1 will have more hit points than enemy 2. Instead we should talk about how the Lazarus Fighter has more armour than Apollo Fighter.

This outline will vary according to the type of game. \\
Opening the game application\\
Game Options \\
Story Synopsis\\
Modes\\
Game Elements\\
Game Levels\\
Player’s Controls\\
Winning\\
Losing\\
End\\
Why is all this fun?\\
\\

\subsection{Nøkkelfunksjoner}

Key features are a list of game elements that are attractive to the player.

It is a good idea to talk about the key features with someone from marketing.\\
Number of Levels\\
Number of Enemies or Characters (Example: 12 characters or amount of enemies, end bosses)\\
Time of Game Play (Example: 2 hours of fun)\\
Replay ability \\
Audio Specifications\\
Graphic Specifications\\
Device Compatibility\\
Number of Players\\
Online Activities (high scores, etc.)\\
Number/Type Modes

\subsection{Selling Features}

Denne delen trenger vi vell ikke å ha med?


