%!TEX root = report.tex
\section{Teknisk utførelse}\label{sec:teknisk}
Denne seksjonen vil detaljere hvordan Garbage Alert per i dag er
implementert, samt tanker om videre utvikling til en eventuell ferdig
versjon.
\subsection{Tidlig prototype}
Tidlig i prosessen ble det utviklet en enkel web-basert prototype som
kunne brukes for å utforske spillets dynamikk. Denne var hjelpsom for å
fastslå hvilke deler av konseptet som fungerte, og ikke fungerte. Se
figur~\ref{fig:screenshot_tidlig_prototype}, under for et skjermskudd av
denne prototypen. Et mål var å ha noe som fungerte, men samtidig
realisere den med minimal bruk av tid.
\begin{figure} [H]
	\begin{center}
	\includegraphics[scale=0.5]{images/screenshot_tidlig_prototype.png}
	\end{center}
	\caption{Skjermskudd av vår tidlige prototype}
	\label{fig:screenshot_tidlig_prototype}
\end{figure}

Denne programkoden, sammen med penn, papir og en solid dose fantasi
gjorde det mulig å ta en spillgjennomgang på et veldig tidlig stadie.
Ved å bruke erfaringer fra denne gjennomgangen var det mulig å justere
spilldynamikken. Det ble mulig å gjøre et bedre anslag på spillets
ressurskostnader og estimere
nedkjølingstider\footnote{Nedkjølingstid er et ord som benyttes om den
tiden det tar fra du utfører en handling til du får lov til å utfører
den igjen.}.

Denne prototype-koden ble senere mye brukt, sammen med en bedre grafisk
presentasjon, for å realisere den endelige protoypen.

\subsection{Endelig prototype}
Garbage Alert-prototypen slik den er i dag er implementert med de nye
web-teknologiene (noe unøyaktig) kjent som HTML5. Mer nøyaktig vil dette
si at spillet er implementert med et \texttt{<canvas>} element, et lite
''malebrett'' som en kan tegne på programmatisk med JavaScript. Dette
medfører at Garbage Alert i prinsippet kan kjøres hvor som helst der en
har en moderne nettleser med støtte for JavaScript. Garbage Alert har
moderne berøringsbaserte mobiltelefoner som hovedplattform, hvor slik
teknologi kjører uten problemer.

Enkelte av elementene er implementert via andre metoder, som for
eksempel hvite overlegg som ser litt malplassert ut i forhold til resten
av grensesnittet. Grunnen til dette er gjenbruk av kode fra en tidligere
prototype. Dette var nødvendig for å raskt kunne skape en prototype som
kunne testes ut og vises fram. I en ferdig versjon ville dette blitt
implementert på en ordentlig måte, konform med resten av koden.


\subsubsection{Systemkrav}
Som nevnt over er Garbage Alert først og fremst ment å kunne spilles på
mobiltelefoner.
Da spillet er implementert som et web-basert spill vil det i prinsippet
ikke være en begrensning på hvilket operativsystem telefonen kjører, det
være Apple iOS, Google Android eller Windows Phone 7, så lenge telefonen
leveres med en moderne nettleser. Nettbrett som Apple iPad og Samsung
Galaxy Tab vil og kunne kjøre Garbage Alert.

I tillegg til dette vil spillet kunne spilles ellers hvor der finnes
moderne/kompatible nettlesere. Det store aberet her er at
kontrollmetodene ikke alltid vil fungere optimalt, og må eventuelt
tilpasses de ulike plattformers styrker og svakheter. Slik prototypen
eksisterer i dag vil den eksempelvis fungere bra med mus og tastatur på
en tradisjonell datamaskin, men ikke i det hele tatt med en styrespak.

Legg dog merke til at ekstensiv testing av ulike nettlesere og enheter
ikke har blitt gjort. Gruppen kan med andre ord ikke gå god for at
spillet kjører feilfritt over alt. Garbage Alert er hovedsaklig testet i
Google Chrome (versjon 18 og over) og Opera (versjon 11.6 og over).
Spillet har blitt testet noe på mobiltelefoner, men ikke ut over det å
sikre at spillet fungerer \emph{greit}. I en endelig versjon må
grensesnittet tilpasses bedre, og ekstensiv testing måtte vært utført
for å sikre at spillet fungerer skikkelig.

\subsubsection{Audiovisuelt innhold}
Garbage Alerts visuelle stil er ment å være leken og samtidig
oversiktlig, og dets grafiske stil er inspirert av den klassiske
spillserien Advance Wars.

Spillets introduksjonsskjerm er ment å være så pompøs og lite subtil som
mulig, med store eksplosjoner på en søppelhaug. På denne skjermen
spilles et utsnitt av sangen \emph{Hell March}, kjent som tittelmusikken
fra Westwoods Red Alert-serie. Dette for å underbygge det pompøse
inntrykket vi ønsker å skape. En ferdig versjon vil inneholde
lydeffekter og bakgrunnsmusikk i resten av spillet.

En mer fullstendig utdypning av spillets visuelle innhold finnes i
seksjon~\ref{sec:artwork}.


\subsubsection{Programmeringsinnhold og kodestruktur}
Som nevnt over er spillet programmert i JavaScript fra grunnen, uten
hjelp av eksisterenede rammeverk. Dette er dels fordi
implementasjonen av spillet er tenkt å være enkel, samt at slike
rammeverk er noe umodne.  Logisk kodestruktur og -arkitektur har ikke
vært noe som har blitt tenkt mye på, og kun enkle grep har blitt gjort
for å gjøre koden noe enklere å skrive. Dette gjelder spesielt den koden
som har blitt arvet fra den tidligere prototypen, kode som med god grunn
kan omtales som et ''lappeteppe''.

For å øke spillets modifiserbarhet og kodens lesbarhet er de ulike
elementene i spillet delt inn i egne JavaScript-filer. Dette for å
logisk samle metoder og funksjoner. I en ferdig versjon vil disse filene
samles i én enkelt fil for å gjøre eksempelvis innlastingstiden bedre.

Spillets bilder og figurer finnes i en egen mappe. På samme måte som med
JavaScript-filene vil en i den ferdige versjonen ha alle bildene i én
større bildefil for å forbedre innlastingstiden.


\subsubsection{Problemer og alternativer}
Siden det er berøringsbaserte mobiltelefoner som er målplattform for
spillet vil de naturlige alternativene være å realisere Garbage Alert
som en såkalt \emph{native} applikasjon\footnote{En \emph{native}
applikasjon er en skreddersydd applikasjon skrevet direkte til ett
enkelt av de ulike mobiltelefonoperativsystemene.}, noe som i
begynnelsen var tanken. Dette ble gått bort fra da ikke alle på gruppen
hadde de nødvendige verktøyene for å utvikle til
plattformen\footnote{Der behøves en Apple Mac og en utviklerlisens, noe
kun ett medlem av gruppa har.}. Da falt valget på å implementere
spille med web-teknologi. Dette førte med seg at flere kunne bidra til
programmeringen både i kraft av at flere har erfaring med JavaScript
fra før og at alle er i besittelse av de verktøy som behøves.

Spill skrevet i JavaScript vil være tregere enn spill skrevet native.
Denne begrensingen vil ikke være et problem for den enkle prototypen.
Senere kan koden skrives om til å kunne kjøres på andre plattformer.

Det er som nevnt ingen garanti at spillet vil fungere i alle nettlesere,
og ekstensiv testing må til for å kunne sikre at det fungerer på flest
mulig enheter. Igjen, grunnet at resultatet av dette prosjektet var ment
å kun være en enkel prototype, er dette noe vi har sett bort fra.

Flerspiller-delen av spillet har ikke blitt implementert, ei heller en
kunstig intelligens til motspilleren.

\subsubsection{Ressurser}
Garbage Alert kan utvikles i hvilken som helst applikasjon som er i
stand til å redigere rene tekstfiler. Kildekontroll gjøres via Git, og
koden er lagret på
GitHub\footnote{\url{https://github.com/aspic/Garbage-Alert}}. Spillet
kan og spilles fra spillet nettside på
\url{http://aspic.github.com/Garbage-Alert}.

\subsubsection{Hvordan spille Garbage Alert lokalt}
Forutsett at du har Git og Python\footnote{Hvilken som helst
server-programvare kan benyttes, Python nevnes her da dette
verktøysettet er temmelig utbredt og er enkelt å starte.} installert,
her er hvordan du starter spillet fra egen datamaskin via en
terminalklient.

Først, last ned kildefilene fra GitHub med kommandoen
\newline\texttt{git clone git://github.com/aspic/Garbage-Alert.git}.
Spillet vil da lastes ned til mappen \texttt{Garbage-Alert}. Navigér til
mappa kalt \texttt{webver} under denne  og start spillet med kommandoen
\texttt{python -m SimpleHTTPServer 8080}. Start en kompatibel nettleser
og navigér til URLen \url{http://localhost:8080}. Garbage Alert vil nå
kjøre lokalt på din datamaskin.
