\section{Sjangeren}\label{sec:sjangre}
Om spillsjangre og sånt.



%Før vi endte opp med sanntidsstrategisjangeren vurderte vi imidlertid en rekke andre mulige sjangre for realiseringen av konseptet. Spesielt eventyr- og rollespillsjangeren ble sterkt vurdert.

Under vil vi beskrive en rekke sjangre som ble vurdert for realiseringen av \emph{Garbage Alert}. Legg dog merke til at navnet ble til en tid etter at sjangeren var valgt.

\subsection{Eventyr}\label{sec:eventyr}
Den første sjangeren som ble vurdert som utgangspunkt for realisering av spillet var eventyrsjangeren. Denne sjangeren er definert ved at en direkte kontrollerer en protagonist i en interaktiv historie. Spilldynamikken består gjerne av det å overvinne ens omgivelser ved utforskning og oppgaveløsning\footnote{Rollings, Andrew; Ernest Adams (2006), Fundamentals of Game Design, Prentice Hall}.

Der finnes en hel rekke undersjangre under eventyrsjangeren. Av spesiell interesse har pek-og-klikk- og action-eventyrsjangrene vært.
	
	\subsubsection{Pek-og-klikk-eventyr}

	Her vil en styre protagonisten ved å peke hvor en vil gå, peke på objekter en vil interagere med og peke på andre mennesker en vil snakke med. Underveis samler en objekter som kan brukes andre steder i spillverden, enten direkte eller gjennom å kombinere gjenstander med hverandre for å skape nye gjenstander som gjør en  stand til å avansere i spillet.

	Populære spill i denne sjangeren er \emph{Lucas Arts'} \emph{Monkey Island}-serien og \emph{Funcoms} \emph{Den Lengste Reisen}.


	\subsubsection{Action adventure}

	Her styrer en tradisjonelt én figur som gradvis finner og samler utstyr som gjør en sterkere og mer motstandsdyktig mot miljø og fiender.


	Populære spill i denne sjangeren er \emph{Nintendos} \emph{The Legend of Zelda} og \emph{Supergiant Games'} \emph{Bastion}.


% \subsection{Rollespill}\label{sec:rollespill}
% RPG



\subsection{Strategi}\label{sec:strategispill}
Den andre sjangeren som ble vurdert for å realisere ideen om et resirkulering- og gjenbruksspill var strategisjangeren. Sentralt i denne sjangeren er ressursforvaltning, noe vi og følte kunne brukes for å realisere et godt konsept.

	\subsubsection{Turbasert strategi}

	Her deles hvert spill inn i \emph{runder}, hvor hver spiller har en egen \emph{tur} til å utføre sine handlinger.


	Den observante leser vil kunne legge merke til at dette ikke er ulikt tradisjonelle brettspill, noe som tilkjennegis ved at mange av de første digitale strategispill var konverteringer av tradisjonelle brettspill.

	Av populære turbaserte strategispill finner vi \emph{Firaxis'} \emph{Sid Meier's Civilization} og \emph{Intelligent Systems'} \emph{Advance Wars}. % TODO: Kilder?



	\subsubsection{Sanntidsstrategi}

	I motsetning til turbaserte strategispill vil ikke hver spiller ha sin egen tur til å utføre sine handlinger, men alle spillere vil utføre sine handlinger samtidig.

	Blant de mer sentrale elementer ved sanntidsstrategispill er konseptene om ressurshåndtering. De fleste spill i denne sjangeren lar det være opp til brukeren å passe på at man til enhver tid har adekvate ressurser til å gjøre den en ønsker og å benytte seg av de rasjonelt. Et annet element som ofte er til stede er stein-saks-papir-balansering av enheter, hvor enhet A kontrer enhet B, B kontrer enhet C og enhet C kontrer enhet A.

	Av populære sanntidsstrategispill finner vi \emph{StarCraft} og \emph{Command \& Conquer}.


\subsection{Vårt valg}
Helt i starten av prosessen som førte til Garbage Alert, startet vi med å diskutere ulike typer eventyr-tilnærminger. Det føltes svært naturlig at i et spill som skal omhandle resirkulering og gjenbruk å låne elementer fra den sjangeren som ofte benytter slike mekanismer. Folk ønsket dog mer fartsfull opplevelse, og diskusjoner om å implementere Zelda-aktige elementer i spillet ble diskutert.

Samtidig ble en alternativ idé utforsket, i kraft av at enkelte ønsket å inkorporere flerspiller og konkurranse som en motivasjonsfaktor. Her startet ideen med å skape et brettspill-inspirert spill. Denne ideen utviklet seg fra å være et svært komplisert turbasert spill til å bli stadig enklere, og stadig raskere, fram til det Garbage Alert til slutt skulle bli; et sanntidsstrategispill.

