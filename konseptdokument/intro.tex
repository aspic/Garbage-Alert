\section{Introduksjon}\label{sec:intro}
Dette avsnittet skal gi et kort overblikk over hvordan resten av
dokumentet er strukturert, hvilke tidligere erfaringer hver av
medlemmene i gruppen har, og hvilke mål som ble satt med tanke på å
utvikle et spillkonsept og en eventuell prototype.

\subsection{Struktur}
Dokumentet er i hovedsak strukturert i 4 hoveddeler:
\begin{enumerate}
	\item Introduksjon (Avsnitt~\ref{sec:intro})
	\item Konsept og spillutforming (Avsnitt~\ref{sec:konsept})
	\item Teknisk (Avsnitt~\ref{sec:teknisk})
	\item Sjanger (Avsnitt~\ref{sec:genre})
	\item Grafisk fremstilling (Avsnitt~\ref{sec:artwork})
	\item Konklusjon (Avsnitt~\ref{sec:conclusion})
\end{enumerate}
Konseptdelen gir et innblikk i hvordan konseptet ble unnfanget og hvilke
valg som lå til grunn for dette. I tillegg gir denne delen en overordnet
forståelse om hva spillet handler om, og går ut på.

Teknisk gir innsikt i selve implementasjonen av prototypen.

\subsection{Gruppemedlemmene}
Herunder vil den spillmessige bakgrunnen til hvert gruppemedlem bli
beskrevet.

\begin{description}
\item[Kjetil Mehl (23), datateknikk] \hfill \\
Kjetil begynte sin spillkarriere med en \gls{Nintendo 64} i 1999. Da
var det eventyr og rollespill (se Avsnitt~\ref{sec:sjangre}) som
\gls{Legend of Zelda} og \gls{Banjo Kazooie} som
stod i fokus. Da datamaskinen kom i hus gikk spillfokuset  over på
konkurransespill som Counter-Strike og \gls{WC3}. Han har i hovedsak vært
en Nintendo~\gls{Fanboy}, men har i de senere år utvidet
kolleksjonen til å omfatte Xbox.

\item[Ina Sander Pedersen (26), bioteknologi] \hfill \\
I 1996 oppdaget Ina \gls{Rayman} til PC og var fra den dag frelst for
serien, men skiftet raskt konsoll til Playstation. For Ina har sjangeren
plattform vært en klar favoritt gjennom årene og da hun senere fikk
\gls{Gameboy} gikk tiden med til \gls{DKC2}. Etterhvert som årene har gått har hun skiftet mellom
Nintendo og Playstation, samt prøvd ut endel forskjellige sjangertyper,
men Rayman og Donkey Kong seriene er og blir favoritter. 

\item[Christian Aleksander Lysne (22), datateknikk, prosjektledelse] \hfill \\
Christian var 8 år da han fikk sin første \gls{Nintendo 64}, med
spillene \gls{Mario 64} og \gls{Mario Kart}.
Senere gikk det i \gls{Legend of Zelda}  og \gls{SSBM}.
\gls{Tekken} var også  populært i oppveksten. \gls{WC3} og
\gls{Diablo 2} tok etter hvert over, og spilles fortsatt i perioder.

\item[Andreas Røysland Aarnes (25), geologi] \hfill \\
Som seksåring prøvde Andreas \gls{Super Mario World} på
\gls{SNES} for første gang, og det var ikke før
i 2011, da New Super Mario Bros. Wii kom til
\gls{Wii}, at han igjen ble tilfredstilt innen
spillverdenen. I mellomtiden har han låst seg til strategispill-seriene
\gls{WC3}, \gls{C and C} og Age of
Empires\cite{aoe}. 

\item[Trond Kjetil (25), informatikk] \hfill \\
Trond Kjetil startet tidlig sin spillkarriere med Nintendo Entertainment System og Super Mario Bros. i 1990, knapt 4 år gammel. TV-spill var fra da av et fast inventar i hans fritidssysler. Senere kom maskiner som Game Boy, Nintendo 64 og PC. I dag spiller han helst førstepersonsskytere på PC, da helst Team Fortress 2 og Tribes Ascend.
\end{description}

\subsection{Mål}
Det overordnede målet for gruppen var å lage et spillkonsept hvert
gruppemedlem følte eierskap til både som et interesseområde og hvor hver person får brukt sin faglige kompetanse. Målet skulle også være konformt med den valgte
problemstillingen (se Avsnitt~\ref{sec:problemstilling}). En prototype i
tillegg til dette ville bli en bonus i forhold til målet.

\subsection{Problemstilling}
Dette avsnittet vil ta for seg den valgte problemstillingen og hva som
lå til grunn for dette valget.
\subsubsection{Valg av problemstilling}
En av de største begrensningene i selve oppgaven var at
problemstillingen måtte sentreres rundt det å skape et interessant spill
om resirkulering og gjenbruk. Etter flere diskusjoner og idèmyldringer ble gruppen
enig om at det var ønskelig å lage et flerspiller-dataspill.
Det ble også ytret ønske om at spillmekanikken skulle foregå både på et
mikroskopisk nivå (lokalt) og makroskopisk nivå (globalt). Ved å
deretter bake inn grunnstenene i problemstillingen (resirkulering og
redusert forbruk), ble problemstillingen forfattet.
\subsubsection{Problemstillingen}\label{sec:problemstilling}
Den valgte problemstillingen er gjengitt nedenfor:
\begin{quotation}
\large\emph{"Hvordan kan en belyse nytten av resirkulering og redusert forbruk,
lokalt og globalt, gjennom flerspiller-dataspill
"}
\end{quotation}
\subsection{Prosessen}
Dette avsnittet forteller hvordan konseptet ble til ved å gå fra en
problemstilling, til en idè og til slutt et konsept. Den første delen av
denne prosessen var ikke iterativ, ettersom gruppen i praksis gikk fra
en vagt definert problemstilling, til en idè, for så å redefinere idèen
for å bedre passe den valgte problemstillingen. Prosessen gikk deretter
fra idè til konsept. 
\subsubsection{Fra problemstilling til idé}
Det ble brukt mye tid på å utforske forskjellige ideer som var
interessante. Gruppen følte seg ikke veldig bundet av den opprinnelige
problemstillingen, og eksperimenterte derfor med forskjellige idèer.
Gitt problemstillingen i avsnitt~\ref{sec:problemstilling} var det
naturlig å tenke på et spill som var turbasert eller sanntidsspill med
flere spillere. Det stod også sentralt å bruke hele gruppens kompentanse. Ettersom tre av gruppens medlemmer har god datakunnskap var det essensielt å flette inn biologi og geologi i ideen. Med fokus rundt dette skapte vi vår første idé.
\begin{description}
\item[Energispillet med flerspillermodus] \hfill\\
Spillere skal organisere hver sin lille by, hvor fokuset er å forhandle
med hverandre på grunnlag av hvilke energitype de forskjellige byene
velger å gå for. En by som for eksempel fokuserer på kull kan raskt få
mye resurser, men ville påvirke sitt eget miljø lokalt, og senere hele
miljøet globalt.
\end{description}
Etter tilbakemeldinger fra landsbylederen viste deg seg derimot at idèen
ikke samsvarte godt nok med problemstillingen som var valgt. Dette førte
til restrukturering av den opphavlige ideen, for å få mer fokus på
gjenbruk og resirkulering. Følgende elementer fra ideen ble tatt med
videre:
\begin{itemize}
	\item Flerspiller
	\item Hver spiller har en base
	\item Bruk av søppel som ressurser, istedenfor energi
	\item Lokale faktorer kan påvirke det globale miljøet
\end{itemize}
\subsubsection{Redefinering av ideen}
Den opprinnelige ideen utviklet seg dermed til et rent konkurransespill, med
mindre fokus på sanking og gjenbruk av resurser. For å minske ressursfokuset
ble det bestemt at hver øy er en "søppeløy", og målet til spilleren er
å fjerne alt søppelet på sin øy.  Ressurser blir utvinnet ved å
resirkulere søppelet på øya. Ved å legge inn denne varianten ble idèen mer
konform med problemstillingen. Energispillet med flerspillermodus hadde en meget stor grad av geoteknologi, geologi, samt bioteknologi. Ettersom fokuset ble vendt mer mot problemstillingen, ble forsvant disse fagområdene fra spillet. Andreas og Ina fant en løsning på dette ved å innføre såkalte geohasarder og biohasarder i spillet (se Avsnitt~\ref{sec:hazards})



\subsection{Wudda!}
% Her: skrive direkte om hva de ulike menneskene i gruppa bidrar med i kraft av sin unike kompetanse.
