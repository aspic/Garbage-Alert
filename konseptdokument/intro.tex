\section{Introduksjon}\label{sec:intro}
Dette avsnittet skal gi et kort overblikk over hvordan resten av
dokumentet er strukturert, hvilke spillmessige bakgrunner hver av
deltagerene i gruppen hadde, og hvilke mål som ble satt med tanke på å
utvikle et spillkonsept og en eventuell prototype.
\subsection{Struktur}
Dokumentet er i hovedsak strukturert i 4 hoveddeler:
\begin{enumerate}
	\item Introduksjon (Avsnitt~\ref{sec:intro})
	\item Konseptet (Avsnitt~\ref{sec:konsept})
	\item Utforming (Avsnitt~\ref{sec:design})
	\item Teknisk (Avsnitt~\ref{sec:teknisk})
\end{enumerate}
Konseptdelen skal gi en overordnet forståelse for hva spillet handler
om. Utformingdelen skal gi dypere innsikt i hvordan spillet er bygd opp,
og hvilke elementer det består av.
\subsection{Gruppemedlemmene}
Herunder vil den spillmessige bakgrunnen til hvert gruppemedlem bli
beskrevet.
\begin{description}
\item[Kjetil Mehl (23)] \hfill \\
Han begynte sin spillkarriere med en \gls{Nintendo 64} i 1999. Da
var det eventyr og rollespill (se Seksjon~\ref{sec:sjangre}) som
\gls{Legend of Zelda} og \gls{Banjo Kazooie} som
stod i fokus. Da datamaskinen kom i hus gikk spillfokuset  over på
konkurransespill som Counter-Strike og \gls{WC3}. Han har i hovedsak vært
en Nintendo~\gls{Fanboy}, men har i de senere år utvidet
kolleksjonen til å omfatte Xbox.
\item[Ina Sander Pedersen (26)] \hfill \\
I 1996 oppdaget hun \gls{Rayman} til PC og var fra den dag frelst for
serien, men skiftet raskt konsoll til Playstation. For Ina har sjangeren
plattform vært en klar favoritt gjennom årene og da hun senere fikk
\gls{Gameboy} gikk tiden med til \gls{DKC2}. Etterhvert som årene har gått har hun skiftet mellom
Nintendo og Playstation, samt prøvd ut endel forskjellige sjangertyper,
men Rayman og Donkey Kong seriene er og blir favoritter. 
\item[Christian Aleksander Lysne (22)] \hfill \\
Christian var 8 år da han fikk sin første \gls{Nintendo 64}, med
spillene \gls{Mario 64} og \gls{Mario Kart}.
Senere gikk det i \gls{Legend of Zelda}  og \gls{SSMB}.
\gls{Tekken} var også  populært i oppveksten. \gls{WC3} og
\gls{Diablo 2} tok etter hvert over, og spilles fortsatt i perioder.
\item[Andreas Røysland Aarnes (25)] \hfill \\
Som seksåring prøvde Andreas \gls{Super Mario World} på
\gls{SNES} for første gang, og det var ikke før
i 2011, da remaken New Super Mario Bros. Wii kom til
\gls{Wii}, at han igjen ble tilfredstilt innen
spillverdenen. I mellomtiden har han låst seg til strategispill-seriene
\gls{WC3}, \gls{C and C} og Age of
Empires\cite{aoe}. 
\end{description}
\subsection{Mål}
Det overordnede målet for gruppen var å lage et spillkonsept hvert
gruppemedlem følte eierskap til, og var konformt med den valgte
problemstillingen (se Seksjon~\ref{sec:problemstilling}). En prototype i
tillegg til dette ville bli en bonus i forhold til målet.
\subsection{Prosessen}
Dette avsnittet forteller hvordan konseptet ble til ved å gå fra en
problemstilling, til en idè og til slutt et konsept. Den første delen av
denne prosessen var ikke iterativ, ettersom gruppen i praksis gikk fra
en vagt definert problemstilling, til en idè, for så å redefinere ideen
for å bedre passe den valgte problemstillingen. Prosessen gikk deretter
fra idè til konsept.
\subsubsection{Fra problemstilling til idè}
Det ble brukt mye tid på å utforske forskjellige ideer som var
interessante. Gruppen følte seg ikke veldig bundet av den opphavlige
problemstillingen, og eksperimenterte derfor med forskjellige ideer.
Gitt problemstillingen i Avsnitt~\ref{sec:problemstillingen} var det
naturlig å tenke på et spill som var turbasert eller sanntidsspill med
flere spillere. Etter flere runder internt ble det enighet rundt
følgende idè.
\begin{description}
\item[Energispillet med flerspillermodus] \hfill\\
Spillere skal organisere hver sin lille by, hvor fokuset er å forhandle
med hverandre på grunnlag av hvilke energitype de forskjellige byene
velger å gå for. En by som for eksempel fokuserer på kull kan raskt få
mye resurser, men ville påvirke sitt eget miljø lokalt, og senere hele
miljøet globalt.
\end{description}
Etter tilbakemeldinger fra landsbylederen viste deg seg derimot at ideen
ikke samsvarte godt nok med problemstillinga som var valgt. Dette førte
til restrukturering av den opphavlige ideen, for å få mer fokus på
gjenbruk og resirkulering. Følgende elementer fra ideen ble tatt med
videre:
\begin{itemize}
	\item Flerspiller
	\item Hver spiller har en base
	\item Bruk av søppel som ressurser, istedenfor energi
	\item Lokale faktorer kan påvirke det globale miljøet
\end{itemize}
\subsubsection{Redefinering av ideen}
Den opphavlige ideen skle over på å bli et rent konkurransespill, med
mindre fokus sanking/utvinning av resurser. For å minske ressursfokuset
ble det bestemt at hver øy er en ``søppeløy'', og målet til spilleren er
å fjerne alt søppelet på sin øy.  Ressurser blir utvinnet ved å
resirkulere søppelet på øya. Ved å legge inn denne vrien ble ideen mer
konformt med problemstillingen. 

%Det ble gjort koblinger opp mot et spill lignende Advanced
%Wars\cite{advancewars} eller Wordfeud\cite{wordfeud} med fokus på
%miljøet.
