\section{Introduksjon}
Dette avsnittet skal gi et kort overblikk over hvordan resten av
dokumentet er strukturert, hvilke spillmessige bakgrunner hver av
deltagerene i gruppen hadde, og hvilke mål som ble satt med tanke på å
utvikle et spillkonsept og en eventuell prototype.
\subsection{Struktur}
Dokumentet er strukturert på en slik måte at utviklingprosessen fra
start til slutt blir beskrevet i kronologisk rekkefølge gjennom
kapitlene i dokumentet. Hele dokumentet kan i hovedsak deles inn i 3
deler:
\begin{enumerate}
	\item Planleggingsfasen
	\item Utviklingsfasen
	\item Ferdigstillingsfasen
\end{enumerate}
Utvikling i denne konteksten går like mye på utvikling av ideen som
utvikling av eventuell kode og produsering av grafikk.
\subsection{Gruppemedlemmene}
Herunder vil den spillmessige bakgrunnen til hvert gruppemedlem bli
beskrevet.
\begin{description}
\item[Kjetil Mehl (23)] \hfill \\
Han begynte sin spillkarriere med en Nintendo 64\cite{n64} i 1999. Da
var det eventyr og rollespill (se Seksjon~\ref{sec:sjangre}) som Legend
of Zelda\cite{legendofzelda} og Banjo Kazooie\cite{banjokazooie} som
stod i fokus. Da datamaskinen kom i hus gikk spillfokuset  over på
konkurransespill som Counter-Strike og Warcraft. Han har i hovedsak vært
en Nintendo fanboy\cite{fanboy}, men har i de senere år utvidet
kolleksjonen til å omfatte Xbox.
\end{description}
\subsection{Mål}
Det overordnede målet for gruppen var å lage et spillkonsept hvert
gruppemedlem følte eierskap til, og var konformt med den valgte
problemstillingen (se Seksjon~\ref{sec:problemstilling}). En prototype i
tillegg til dette ville bli en bonus i forhold til målet.
