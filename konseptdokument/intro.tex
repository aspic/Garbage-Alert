\section{Introduksjon}
Dette avsnittet skal gi et kort overblikk over hvordan resten av
dokumentet er strukturert, hvilke spillmessige bakgrunner hver av
deltagerene i gruppen hadde, og hvilke mål som ble satt med tanke på å
utvikle et spillkonsept og en eventuell prototype.
\subsection{Struktur}
Dokumentet er strukturert på en slik måte at utviklingprosessen fra
start til slutt blir beskrevet i kronologisk rekkefølge gjennom
kapitlene i dokumentet. Hele dokumentet kan i hovedsak deles inn i 3
deler:
\begin{enumerate}
	\item Planleggingsfasen
	\item Utviklingsfasen
	\item Ferdigstillingsfasen
\end{enumerate}
Utvikling i denne konteksten går like mye på utvikling av ideen som
utvikling av eventuell kode og produsering av grafikk.
\subsection{Gruppemedlemmene}
Herunder vil den spillmessige bakgrunnen til hvert gruppemedlem bli
beskrevet.
\begin{description}
\item[Kjetil Mehl (23)] \hfill \\
Han begynte sin spillkarriere med en Nintendo 64\cite{n64} i 1999. Da
var det eventyr og rollespill (se Seksjon~\ref{sec:sjangre}) som Legend
of Zelda\cite{legendofzelda} og Banjo Kazooie\cite{banjokazooie} som
stod i fokus. Da datamaskinen kom i hus gikk spillfokuset  over på
konkurransespill som Counter-Strike og Warcraft. Han har i hovedsak vært
en Nintendo fanboy\cite{fanboy}, men har i de senere år utvidet
kolleksjonen til å omfatte Xbox.
\item[Ina Sander Pedersen (26)] \hfill \\
I 1996 oppdaget hun Rayman\cite{rayman} til PC og var fra den dag frelst for serien, men skiftet raskt konsoll til Playstation. For Ina har sjangeren plattform vært en klar favoritt gjennom årene og da hun senere fikk Gameboy Pocket\cite{gameboy} gikk tiden med til Donkey Kong Country 2\cite{DKC2}. Etterhvert som årene har gått har hun skiftet mellom Nintendo og Playstation, samt prøvd ut endel forskjellige sjangertyper, men Rayman og Donkey Kong seriene er og blir favoritter. 
\item[Christian Aleksander Lysne (22)] \hfill \\
Christian var 8 år da han fikk sin første Nintendo 64\cite{n64} , med spillene Super Mario 64\cite{mario64} og Mario Kart\cite{mariokart}. Senere gikk det i Legend of Zelda\cite{legendofzelda} og Super Smash Bros\cite{smash}. Tekken\cite{tekken} var også  populært i oppveksten. Warcraft 3\cite{wc3} og Diablo 2\cite{diablo2} tok etter hvert over, og spilles fortsatt i perioder.
\end{description}
\subsection{Mål}
Det overordnede målet for gruppen var å lage et spillkonsept hvert
gruppemedlem følte eierskap til, og var konformt med den valgte
problemstillingen (se Seksjon~\ref{sec:problemstilling}). En prototype i
tillegg til dette ville bli en bonus i forhold til målet.
