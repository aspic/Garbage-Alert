%!TEX root = report.tex
\section{Introduksjon}\label{sec:intro}
Dette avsnittet skal gi et kort overblikk over hvordan resten av
dokumentet er strukturert, hvilke tidligere erfaringer hver av
medlemmene i gruppen har, og hvilke mål som ble satt med tanke på å
utvikle et spillkonsept og en eventuell prototype.

\subsection{Struktur}
Dokumentet er i hovedsak strukturert i tre deler:
\begin{enumerate}
	\item Konsept og spillutforming (seksjon~\ref{sec:konsept})
	\item Teknisk (seksjon~\ref{sec:teknisk})
	\item Grafisk fremstilling (seksjon~\ref{sec:artwork})
\end{enumerate}
Konsept og spillutforming gir et innblikk i hvordan konseptet ble unnfanget og hvilke
valg som lå til grunn for dette. I tillegg gir denne delen en overordnet
forståelse om hva spillet handler om, og går ut på.

Den tekniske delen gir innsikt i selve implementasjonen av prototypen, og den grafiske fremstillingen
gir en enkel illustrasjon av spillet.

\subsection{Gruppemedlemmene}
Herunder vil den spillmessige bakgrunnen til hvert gruppemedlem bli
beskrevet.

\begin{description}
\item[Kjetil Mehl (23), datateknikk] \hfill \\
Kjetil begynte sin spillkarriere med en \gls{Nintendo 64} i 1999. Da
var det eventyr (se seksjon~\ref{sec:genre}) og rollespill som
\gls{Legend of Zelda} og \gls{Banjo Kazooie} som
stod i fokus. Da datamaskinen kom i hus gikk spillfokuset  over på
konkurransespill som \gls{Counter-Strike} og \gls{Warcraft III}. Han har i hovedsak vært
en Nintendo~\gls{Fanboy}, men har i de senere år utvidet
kolleksjonen til å omfatte Xbox.

\item[Ina Sander Pedersen (26), bioteknologi] \hfill \\
I 1996 oppdaget Ina spillet \gls{Rayman} til PC og var fra den dag frelst for
serien, men skiftet raskt konsoll til Playstation. For Ina har sjangeren
plattform vært en klar favoritt gjennom årene og da hun senere fikk
\gls{Game Boy} gikk tiden med til \gls{Donkey Kong Country 2}. Etterhvert som årene har gått har hun skiftet mellom
Nintendo og Playstation, samt prøvd ut endel forskjellige sjangertyper,
men Rayman- og Donkey Kong-seriene er og blir favoritter. 

\item[Christian Aleksander Lysne (22), datateknikk, prosjektledelse] \hfill \\
Christian var 8 år da han fikk sin første \gls{Nintendo 64}, med
spillene \gls{Mario 64} og \gls{Mario Kart}.
Senere gikk det i \gls{Legend of Zelda}  og \gls{Super Smash Brothers}.
\gls{Tekken} var også  populært i oppveksten. \gls{Warcraft III} og
\gls{Diablo 2} tok etterhvert over, og spilles fortsatt i perioder.

\item[Andreas Røysland Aarnes (25), geologi] \hfill \\
Som seksåring prøvde Andreas \gls{Super Mario World} på
\gls{SNES} for første gang, og det var ikke før
i 2011, da New Super Mario Bros. kom til
\gls{Wii}, at han igjen ble tilfredstilt innen
spillverdenen. I mellomtiden har han låst seg til strategispill-seriene
\gls{Warcraft III}, \gls{Command and Conquer} og Age of
Empires. 

\item[Trond Kjetil (25), informatikk] \hfill \\
Trond Kjetil startet tidlig sin spillkarriere med \gls{NES} og Super Mario Bros. i 1990, knapt 4 år gammel. TV-spill var fra da av et fast inventar i hans fritidssysler. Senere kom maskiner som \gls{Game Boy}, \gls{Nintendo 64} og PC. I dag spiller han helst førstepersonsskytere på PC, da helst \gls{Team Fortress 2} og \gls{Tribes Ascend}.
\end{description}

\subsection{Gruppens kompetanse}

Kjetil, Trond Kjetil og Christian er mennesker med bakgrunn fra datafaglig utdanning, henholdsvis datateknikk og informatikk fra NTNU og dataingeniør fra HiST. Disse tok derfor på seg oppgaven relatert til den tekniske implementasjonen av spillet.
Ina har sin bakgrunn i bioteknologi fra NTNU, og Andreas' bakgrunn stammer fra petroleumsgeologi, også fra NTNU.
Deres bakgrunn og faglige kompetanse kommer til syne i spillelementer integrert i spillet.
Christian studerer i dag Project Management, og gruppen kunne derfor ha benyttet seg av hans prosjektstyringskunnskaper. Da det har blitt valgt å holde gruppestrukturen så flat som mulig, har det blitt lagt mest vekt på hans databakgrunn, hvor hovedkompetansen hans ligger.

\subsection{Mål}
Det overordnede målet for gruppen var å lage et spillkonsept hvert
gruppemedlem følte eierskap til både som et interesseområde og hvor hver person får brukt sin faglige kompetanse. Målet skulle også være konformt med den valgte
problemstillingen (se seksjon~\ref{sec:problemstilling}). Det å kunne produsere en fungerende prototype i tillegg til dette ville bli en bonus i forhold til målet.

\subsection{Problemstilling}
\label{sec:problemstilling}
Denne seksjonen vil ta for seg den valgte problemstillingen og hva som
lå til grunn for dette valget.
\subsubsection{Valg av problemstilling}
En av de største begrensningene i selve oppgaven var at
problemstillingen måtte sentreres rundt det å skape et interessant spill
om resirkulering og gjenbruk. Etter flere diskusjoner og idémyldringer ble gruppen
enig om at det var ønskelig å lage et flerspiller-dataspill.
Det ble også ytret ønske om at spillmekanikken skulle foregå både på et
mikroskopisk nivå (lokalt) og et makroskopisk nivå (globalt). Ved å
deretter bake inn grunnstenene — resirkulering og
gjenbruk — ble problemstillingen forfattet.
\subsubsection{Problemstillingen}\label{sec:problemstilling}
Den valgte problemstillingen er gjengitt i sin helhet nedenfor.
\begin{quotation}
\large\emph{"Hvordan kan en belyse nytten av resirkulering og redusert forbruk,
lokalt og globalt, gjennom flerspiller-dataspill
"}
\end{quotation}
Denne problemstillingen gjør det mulig å flette inn kompetansen til de
ulike medlemmene i spillet. De som har studert data får brukt
kompetansen sin på et overordnet nivå når det gjelder selve utviklingen.
Ved å ha med ''gjenbruk og resirkulering'' kan en lage et spill med
grunnlag i biologiske og kjemiske prosesser. I tillegg åpner det opp for å bruke geologi ved å inkludere spillementer basert på mineralogi og geologiske hendelser. 

\subsection{Prosessen}
Denne seksjonen forteller hvordan konseptet ble til ved å gå fra en problemstilling, til en idé og til slutt et konsept. For utvikling og realisering av konsepte ble det benyttet en iterativ og smidig arbeidsmetode \cite{online:agile_manifesto} basert på Scrum-metodikken \cite{Scrum}. Her vil en dele opp arbeidet i mindre, mer håndterlige deler en raskt kan fullføre, for videre å teste ut disse elementene for å se hva som fungerer bra og hva som fungerer mindre bra. I praksis vil dette si at gruppen raskt skapte en spillbar prototype hvor ting ble prøvd ut, for så å iterativt utvide denne til stadig mer avanserte prototyper basert på erfaringene gruppen gjorde seg underveis.

Den største fordelen ved å gjøre det på denne måten er at en raskt kan justere ens kurs og retning basert på erfaringer man gjør seg opp. 
Den største bakdelen med denne metodikken er at en raskt skaper en del kode og gjør en del arbeid for å skape en enkel prototype som ikke kan brukes videre.
I forhold til dette faget hvor målet var å produsere en enkel prototype var denne metodikken svært passende, spesielt da det er mulig å hele tiden er mulig å fryse utviklingen når prototypen er ''avansert nok''.


En enkelt Gantt-diagram (se appendiks \ref{app:Gantt}) ble produsert for å prøve å ha en viss oversikt over hva som burde og hva som må være ferdig til gitte tider, samt overslag over hvor lang tid ting vil ta.
Morgenmøter ble holdt på starten av hver onsdag, hvor gruppens medlemmer forteller om hva de har gjort siden sist og hva de planlegger å gjøre fram til neste morgenmøte. Dette for å sørge for at alle til enhver tid var oppdaterte på hva som skjer og må gjøres framover.
Allikevel, på grunn av at gruppen til enhver tid jobbet tett sammen, var det en jevn flyt av informasjon angående hva som skjer gjennom hele dagen.


Den første delen av prosessen var allikevel ikke utført iterativ, ettersom gruppen i praksis gikk fra en vagt definert problemstilling, til en idé, for så å redefinere ideen for å bedre passe den valgte problemstillingen. Prosessen gikk deretter fra idé til konsept.
\subsubsection{Fra problemstilling til idé}
Det ble brukt mye tid på å utforske forskjellige ideer som var
interessante. Gruppen følte seg ikke veldig bundet av den opprinnelige
problemstillingen, og eksperimenterte derfor med forskjellige ideer.
Gitt problemstillingen i seksjon~\ref{sec:problemstilling} var det
naturlig å tenke på et spill som var turbasert eller sanntidsspill med
flere spillere. Det stod også sentralt å bruke hele gruppens kompentanse. Ettersom tre av gruppens medlemmer har god datakunnskap var det essensielt å flette inn biologi og geologi i konseptet. Med fokus rundt dette ble den første ideen skapt.
\begin{description}
\item[Energispillet med flerspillermodus] \hfill\\
Spillere skal organisere hver sin lille by, hvor fokuset er å forhandle
med hverandre på grunnlag av hvilke energitype de forskjellige byene
velger å gå for. En by som for eksempel fokuserer på kull kan raskt få
mye ressurser, men ville påvirke sitt eget miljø lokalt, og senere hele
miljøet globalt.
\end{description}
Etter tilbakemeldinger fra landsbylederen viste deg seg derimot at ideen
ikke samsvarte godt nok med problemstillingen som var valgt. Dette førte
til restrukturering av den opphavlige ideen, for å få mer fokus på
gjenbruk og resirkulering. Følgende elementer fra ideen ble tatt med
videre:
\begin{itemize}
	\item Flerspiller
	\item Hver spiller har en base
	\item Bruk av søppel som ressurser, istedenfor energi
	\item Lokale faktorer kan påvirke det globale miljøet
\end{itemize}
\subsubsection{Redefinering av ideen}
Den opprinnelige ideen utviklet seg dermed til et rent konkurransespill,
med mindre fokus på sanking og utvinning, og mer fokus på gjenbruk av ressurser. 
Det ble bestemt at hver øy skal starte som en "søppeløy", og målet til
spilleren er å fjerne alt søppelet på sin øy.  Ressurser blir anskaffet
ved å resirkulere søppelet på øya. Ved å legge inn denne varianten ble
ideen mer konform med problemstillingen. Energispillet med
flerspillermodus hadde en meget stor grad av geoteknologi og
bioteknologi. Ettersom fokuset i spillet ble endret,
forsvant disse sentrale fagområdene. Andreas og Ina fant en løsning
på dette ved å innføre såkalte geohasarder og biohasarder i spillet (se
seksjon~\ref{sec:hazards})



