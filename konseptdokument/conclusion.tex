\section{Resultat}\label{sec:conclusion}
Denne seksjonen vil gi en kort oppsumering av konseptet, diskutere spillets samfunnsnytte og forklare hva som er tenkt rundt videre arbeid.

\subsection{Oppsummering}
Garbage Alert er spillet der du bestemmer verdens skjebne gjennom å gjenvinne og krige med med søppel. Gjennom å oppgradere sin miljøstasjon får man tilgang til forbedrede og nye metoder rundt gjenvinning, noe som gir spilleren muligheten til å avansere og til slutt gå av med seieren.  Geologiske og biologiske katastrofale hendelser oppstår som konsekvenser av krig med søppel og dårlige valg rundt gjenvinning. 
Spillet er utviklet av tre datastudenter, en geologistudent og en biologistudent ved NTNU. Deres tverrfaglighet og ulike egenskaper har gjennom en iterativ prosess skapt et konsept som de mener kan endre holdninger, opplyse om resirkulering, gjenvinning og dets konsekvenser gjennom et flerspiller-strategispill som sikter seg mot barn og unge mellom 10 og 17 år. 
Resultatet av deres arbeid er et spillkonsept og en prototype for mobile enheter. Konseptet tilrettelegger for muligheten for å danne en betaversjon av et underholdende og spennende spill med et morsomt audiovisuelt innhold. Adrenalinet av å være i krig med andre spillere, raske og ulike taktikker som overrasker de andre spillerne og plutselige katastrofer som kan snu spillet i alle retninger, er elementer som gjør Garbage Alert til den neste mobilapplikasjonen alle snakker om. 


\subsection{Spillets samfunnsnytte}
Garbage Alert har som mål å belyse nytten av gjenbruk og resirkulering, lokalt og globalt, gjennom et flerspiller dataspill (se seksjon \ref{sec:problemstilling}).

Gjenvinning av søppel står sentralt i Garbage Alert. Når spilleren gjenvinner ulike ressurser fra søppelet vil han/hun se at man ikke nødvendigvis er avhengig av å utvinne nye ressurser, men at man kan bruke ressursene man har på ny. Dette vil forhåpentligvis gi økt bevissthet rundt gjenbruk.

De ulike oppgraderingene som finnes i spillet kan relateres til forskning innen gjenbruk og resirkulering, som gjør det mulig å gjenbruke flere typer ressurser fra søppelet samt øke gjenvinningsgraden og effektiviteten. Dette har som mål å belyse hvor viktig det er å forske på disse aspektene rundt gjenbruk og nytten vi har av det.

Krig, som står sentralt i Garbage Alert, er ikke direkte en samfunnsnyttig del av spillet. Det er derimot flere aspekter med dette som kan relateres til dagens samfunn. De forskjellige mulighetene spilleren har innen bygging av forsvar og angrep viser at et økt gjenbruk (som ved å gi spilleren flere ressurser) gir muligheten til å bygge bedre bygninger, noe som må til for å vinne spillet. I tillegg til dette har katastrofene som kan inntreffe i spillet en parallell til dagens samfunn. Ved å ikke ta ansvar for søppelet, øker sannsynligheten for globale katastrofer, der global oppvarmning er et meget viktig tema idag. Noen av de geologiske katastrofene i spillet kan være et direkte resultat av global oppvarmning. De biologiske katastrofene er knyttet til mulige utfall dersom man ikke blir kvitt søppelet. I Garbage Alert finnes det både lokale og globale katastrofer, noe som er ment å belyse at både deg selv og andre blir berørt av handlinger som ikke nødvendigvis er utført av deg.

Katastrofene i spillet er i større grad en eksplisitt informativ konsekvens som belyser negative følger av dårlig håndtering av søppel. De fleste informative delene av spillet er derimot implisitte, og vil forhåpentligvis øke spillerens bevisshet rundt økt gjenbruk og nytten av dette. 

\subsection{Videre arbeid}
Et naturlig videre arbeid med Garbage Alert vil være å fullføre prototypen med implementasjon av flerspillerfunksjonalitet og ferdigstilt grafikk på mobile enheter. For å kunne fullføre prototypen burde minimum en designer og en programmerer ta del i prosjektet for å oppnå den samlede kompetansen som kreves. Ved ferdigstillelse av prototypen vil det være aktuelt å finne samarbeidspartnere som kan bidra med finansiering og markedsføring.
