\section{Avslutning}\label{sec:conclusion}
Denne seksjonen vil relatere prosjektet i forhold til spilldesign som
felt og diskutere spillets samfunnsnytte og forklare hva som er tenkt
rundt videre arbeid.

\subsection{Garbarge Alert, i en større sammenheng}
Spilldesign på et generelt nivå starter ofte med en idè, eller en
modifisering av et eksisterende konsept. Prosessen videre varierer
veldig fra utvikler til utvikler og selskap til selskap, men en del
fellestrekk finnes:
\begin{itemize}
\item Definere spillbare elementer
\item Nivådesign
\item Innhold
\item Mekanikk
\end{itemize}
Etterhvert som denne prosessen går videre er det vanlig å dokumentere
alt i et designdokument. Dette dokumentet kan bli brukt som et rent
oppslagsverk for utviklere, og vil konstant være i endring. En felles
arbeidsflate for dette (en egen wiki eller lignende) blir også ofte brukt.

Måten Re³ designet Garbarge Alert på, har klare trekk med den generelle
spilldesign-metoden.


\subsection{Spillets samfunnsnytte}
Garbage Alert har som mål å belyse nytten av gjenbruk og resirkulering, lokalt og globalt, gjennom et flerspiller dataspill (se seksjon \ref{sec:problemstilling}).

Gjenvinning av søppel står sentralt i Garbage Alert. Når spilleren gjenvinner ulike ressurser fra søppelet vil han/hun se at man ikke nødvendigvis er avhengig av å utvinne nye ressurser, men at man kan bruke ressursene man har på ny. Dette vil forhåpentligvis gi økt bevissthet rundt gjenbruk.

De ulike oppgraderingene som finnes i spillet kan relateres til forskning innen gjenbruk og resirkulering, som gjør det mulig å gjenbruke flere typer ressurser fra søppelet samt øke gjenvinningsgraden og effektiviteten. Dette har som mål å belyse hvor viktig det er å forske på disse aspektene rundt gjenbruk og nytten vi har av det.

Krig, som står sentralt i Garbage Alert, er ikke direkte en samfunnsnyttig del av spillet. Det er derimot flere aspekter med dette som kan relateres til dagens samfunn. De forskjellige mulighetene spilleren har innen bygging av forsvar og angrep viser at et økt gjenbruk (som ved å gi spilleren flere ressurser) gir muligheten til å bygge bedre bygninger, noe som må til for å vinne spillet. I tillegg til dette har katastrofene som kan inntreffe i spillet en parallell til dagens samfunn. Ved å ikke ta ansvar for søppelet, øker sannsynligheten for globale katastrofer, der global oppvarmning er et meget viktig tema idag. Noen av de geologiske katastrofene i spillet kan være et direkte resultat av global oppvarmning. De biologiske katastrofene er knyttet til mulige utfall dersom man ikke blir kvitt søppelet. I Garbage Alert finnes det både lokale og globale katastrofer, noe som er ment å belyse at både deg selv og andre blir berørt av handlinger som ikke nødvendigvis er utført av deg.

Katastrofene i spillet er i større grad en eksplisitt informativ konsekvens som belyser negative følger av dårlig håndtering av søppel. De fleste informative delene av spillet er derimot implisitte, og vil forhåpentligvis øke spillerens bevisshet rundt økt gjenbruk og nytten av dette. 

\subsection{Videre arbeid}
Et naturlig videre arbeid med Garbage Alert vil være å fullføre prototypen med implementasjon av flerspillerfunksjonalitet og ferdigstilt grafikk på mobile enheter. For å kunne fullføre prototypen burde minimum en designer og en programmerer ta del i prosjektet for å oppnå den samlede kompetansen som kreves. Ved ferdigstillelse av prototypen vil det være aktuelt å finne samarbeidspartnere som kan bidra med finansiering og markedsføring.
