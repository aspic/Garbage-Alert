\section{Avslutning}

I løpet av semesteret med EiT synes vi selv at vi har hatt stor utviklig som gruppe, og at vi alle er blitt mer bevisst på både
vår egen og andres rolle i ulike situasjoner. Gruppen er enig om at EiT har gitt mange erfaringer
som hver enkelt av oss kan bringe videre, både til bruk i resten av studitiden og ikke minst i arbeidslivet. 


Ut i fra tilbakemeldinger fra facilitator har vi som medlemmer i gruppen blitt mer bevisst på kommunikasjonsmønsteret 
gjennom hele prosessen og utviklet evnen til å se hvilke endringer som kan gjøres for å flytte på et mønster 
som ikke er tilfredsstillende. Vi har fått verdifulle 
verktøy om hvordan en selv kan hjelpe gruppen framover med enkle grep og metoder, der det å innta 
roller som man ser at gruppen kan dra nytte av ligger sentralt.

I gjennom tiden med EiT har alle gruppens medlemmer måtte tenke over og bli bevisst på sin egen og andres rolle
i gruppen, noe som har gjort at vi både som indivder og sammen som gruppe har hatt en jevn framgang og læringskurve.

Fra øvelsen om Person-Gruppe Relasjon lærte gruppen at en ikke nødvendigvis har samme tanker om hvordan en selv
oppfattes i en situasjon som det resten av gruppen har, og De Bonos Hatter har gitt gruppen et sterkt analyseverktøy 
der det å gi tilbakemelding på spesifikke
områder kan ufarliggjøres ved å ta på seg forskjellige roller. 

Hvert enkelt av gruppemedlemmene har gjort følgende tanker om hva nettopp de vil ta med seg videre fra semesteret:
\begin{description}
\item[Andreas] \hfill \\

\item[Christian] \hfill \\

\item[Ina] \hfill \\
Ina tar med seg en bedre innstilling til framtidig gruppearbeid og troen på at hvis man jobber for 
det vil man kunne skape en gruppe med god dialog og et godt samspill. Hun føler at hun har lært mye fra de andre
i gruppen både om nye dataprogrammer hun gjerne vil bruke i framtiden, og om seg selv som person.
Ina er også blitt mer bevisst på hvordan hun oppfører seg i en gruppesituasjon og ser på semesteret med
EiT som et svært lærerikt halvår.

\item[Kjetil] \hfill \\

\item[Trond] \hfill \\

\end{description}
Alle i gruppen føler at de gjennom prosessen har blitt hørt slik at hvert medlem føler eierskap og er stolt av det ferdige produktet. 
En viktig grunn til dette er at gruppen gjennom hele prosessen har tatt hensyn til at medlemmene har forskjellig faglig bakgrunn og 
at sluttproduktet skulle være et resultat av tverrfagligheten.


