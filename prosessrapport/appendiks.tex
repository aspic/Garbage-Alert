\appendix
\addcontentsline{toc}{section}{Appendiks}
\addtocontents{toc}{\protect\setcounter{tocdepth}{-1}}
\section{Samarbeidsavtale} \label{A} 
\subsection{Hovedmomenter}
Gruppa må prøve å finne oppgaver, samt fordele disse. Det er viktig at gruppa setter felles mål, og at medlemmene er enige i desse. Gruppa må være åpen for at alle kan komme med innspill. Ingen bidrag er for dårlige. Det er lov å diskutere, eventuelt krangle litt.
\subsection{Oppmøtetider}
Gruppemedlemmene er individuelt ansvarlige for å møte opp kl. 0815. Gitt at noe spesielt skjer (personer er hindret fra å møte), gi beskjed. Mailinglista er til for å brukes. Oppmøtetider utenom bør gjerne starte litt senere enn 0815. Sanksjoner kan bli gitt, om oppmøtetider viser seg å være vanskelige å holde.
\subsection{Møtestruktur}
Begynner med en runde om hva hver enkelt har gjort før møtet, samt en runde hvor vi forteller hva planen er frem til neste møte. Her er det viktig å fortelle om problemer som har, eller kan oppstå. Møtene tar vi ad-hoc. Det skal være en møteleder/referent som fører møtelogg.
\subsection{Innleveringsfrister}
Vi må sette milepæler underveis, gjerne vist grafisk. Frister for rapporter må overholdes, men vi tar sikte på å bli ferdig i god tid før endelig frist. Milepæler kan endres ift. framgang (under konstant vurdering).
\subsection{Arbeidssteder}
Gitt at vi kan, møtes vi på Gløshaugen. Om det er lagt opp til morningsmøte/avslutningsmøte på Dragvoll, må vi føye oss etter dette. Vi bør bestille grupperom avhengig av hvor vi skal møtes, dette bør bli gjort i god tid før “arbeidsdag”.
\subsection{Kommunikasjonsform}
Alt skriftlig arbeid (sett bortifra kode) skal bli gjort i “Google docs”. Leveringer må bli gjort på “its learning”. Fjernkommunikasjon vil bli gjort gjennom mailingliste.
\subsection{Ledelsesfunksjoner}
Flat struktur. En ansvarlig for gruppemøte, kalla “dagens referent”.
\subsection{Beslutningsstrategi}
Er det noe vi absolutt ikke blir enige om, vil flertallet bestemme. Om personer ikke er til stede vil de til en viss grad miste “stemmerett”. Er det derimot en stor beslutning, bør alle stemmer høres.
\subsection{Ambisjonsnivå}
Gruppa er enig om at vi jobber for en A.